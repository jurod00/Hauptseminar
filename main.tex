% Link to this project: https://www.overleaf.com/read/hdpkgxgjkbgw

%% TeXstudio and TeXShop accept magic comments
% !TeX encoding = utf8
% !TeX spellcheck = en_GB
%% !BIB program = biber

\RequirePackage[l2tabu, orthodox]{nag} % at beginning to track errors
\documentclass[a4paper, english, 11pt, onecolumn, oneside, final]{article}

\usepackage[T1]{fontenc}        % Use 8-bit encoding that has 256 glyphs, for output
\usepackage[utf8]{inputenc}     % to input ä, ß, ø, æ, å, „“, “” etc. (incl. bib)
\usepackage[verbose, a4paper, hcentering]{geometry}	% customize page layout, total={418.3pt, 591.5pt}, height=.9\paperheight

\geometry{a4paper}
%\usepackage{color}	            % enables colors
\usepackage[english]{babel}     % English (as References, e.g.)
\usepackage{amsthm, mathtools, amssymb, nicefrac} % mathtools loads amsmath
\usepackage{algorithm2e}        % enables algorithms
%\mathtoolsset{showonlyrefs}
\numberwithin{equation}{section}   % numbering equations

%   ╭════════════════════════════════════
%	│	Developers tools
\usepackage[notcite,notref, draft]{showkeys}	%  display labels and bookmarks
\setlength {\marginparwidth }{2cm}  % just avoid a package warning
\usepackage[textsize= tiny, obeyFinal]{todonotes}  %  enable \todo{...}


%	╭───────────────────────────────────────────────
%	│	User preferential fonts
%\PassOptionsToPackage{full}{textcomp}
\usepackage{newtxtext, newtxmath}%	replaces Springer like postscript (PS) fonts (Times)
%\usepackage{lmodern}           % improved pdf fonts
\usepackage{dsfont}             % für \one und Erwartungswert, doublestroke
\usepackage{eurosym}
\usepackage{microtype}          % after fonts, for typographical perfection

%	╭───────────────────────────────────────────────
%	│	tables and figures
\usepackage{wrapfig}    % enables figures floating in the text
\usepackage{caption, subcaption} % replaces deprecated subfig
%\usepackage{graphicx}  % tikz loads this package as well
%\setkeys{Gin}{draft}   % draft= false
\usepackage{array}      % extends array and tabular environments
\usepackage{booktabs}   % more beautiful tables
\usepackage{tikz}       % nice tool for professional figures

%	╭───────────────────────────────────────────────
%	│	references and enumerations
\usepackage{siunitx}    % detect-all; [locale=DE] for German commas, etc
\DeclareSIUnit{\sieuro}{\mbox{\euro}}   % provide € symbol in formulas


\usepackage[inline, shortlabels]{enumitem}           % customized list environments, as roman
\setlist[enumerate,1]{label=(\roman*)}  % 1st layer (i)
\usepackage[numbers]{natbib}	% References: round, numbers
\usepackage[bookmarks=true, 
    pdfpagemode= UseNone, 
    backref= page, % include back-reference in references
    pdftitle={Scientific writing}, 
    pdfauthor={Alois Pichler}, 
    colorlinks=true, citecolor=ForestGreen, urlcolor=darkgray, linkcolor=ForestGreen, 
    pdfstartview= FitH]{hyperref} % load as last package
\usepackage{doi}                % provides proper doi-links in references: load after hyperref
\definecolor{ForestGreen}{HTML}{228B22} %005f50

\usepackage[capitalise]{cleveref}   % lazy referencing
\usepackage[all,defaultlines=2]{nowidow} % Hurenkinder, Schusterjungen

%	╭───────────────────────────────────────────────
%	│	User specified commands
\definecolor{TUC}{RGB}{0, 95, 80}%\normalcolor
\definecolor{Alois}{RGB}{161, 11, 112}
%%   response letter
\newenvironment{response}[1][blue]{\par\color{#1}\vspace{0.4em}\itshape\textbf{Response: }}{\par}			% response
%%   revision
\newcommand{\rev}[2][blue]{{\color{#1}#2}}       % command
\newenvironment{revise}[1][blue]{\color{#1}}{}   % environment

%%   my commands

\DeclareMathOperator{\var}{\mathrm{var}}    % variance
\DeclareMathOperator{\cov}{\mathrm{cov}}	% covariance
\DeclareMathOperator{\pr}{\mathrm{Pr}}     % Probability
\DeclareMathOperator{\E}{\mathrm{E}}	    % Expectation
\DeclareMathOperator*{\essinf}{ess\,inf}    % essential supremum
\DeclareMathOperator*{\esssup}{ess\,sup}    %
\DeclareMathOperator*{\argmax}{arg\,max}
\DeclareMathOperator{\VaR}{\mathrm{VaR}}	% Value-at-Risk
\DeclareMathOperator{\AVaR}{\mathrm{AVaR}}	% Average Value-at-Risk

\newcommand{\R}{\mathbb{R}}                 % real numbers
\newcommand{\one}{\mathds 1}
\newcommand{\precision}{\Sigma^{-1}}
\newcommand{\A}{r^\top\Sigma^{-1} r}
\newcommand{\B}{r^\top\Sigma^{-1}\one}
\newcommand{\C}{\one^\top\Sigma^{-1}\one}
\newcommand{\D}{(\A)(\C)-(\B)^2}

%%%%%%%%%%%%%%%%%%%%%%%%% the "\LyX" command.
\providecommand{\LyX}{\texorpdfstring{L\kern-.1667em\lower.25em\hbox{Y}\kern-.125emX\@}{LyX}}


%	╭───────────────────────────────────────────────
%	│	math environments for Theorems, Definitions, Remarks etc.
\theoremstyle{plain}
	\newtheorem{theorem}             {Theorem}[section]
	\newtheorem{corollary}  [theorem]{Corollary}
	\newtheorem{lemma}      [theorem]{Lemma}
	\newtheorem{proposition}[theorem]{Proposition}
\theoremstyle{definition}
	\newtheorem{definition} [theorem]{Definition}
\theoremstyle{remark}
	\newtheorem{remark}     [theorem]{Remark}

%	╭───────────────────────────────────────────────
%   │     the main document
\title{
    Sensitivity of Optimal Solutions to the Markowitz Portfolio Optimization Problem under certain Risk Measures
}

\author{
    Julius Rode%
        \thanks{Chemnitz, University of Technology. Contact: \href{mailto:julius.rode@s2018.tu-chemnitz.de}{julius.rode@s2018.tu-chemnitz.de}}
}

%  █▀▙ █▀█ █▀▀ █░█ ▙▃▟ █▀▀ █▖█ ▀█▀
%  █▄▛ █▄█ █▄▄ █▄█ █▔█ ██▄ █▝█ ░█░
\begin{document}
\maketitle

\begin{abstract}
    This primer provides some basic and elementary hints to deliver a final thesis or an article.
    The primer itself is written in \LaTeX{} via \href{www.overleaf.com}{overleaf} so that source code can simply be copy/\,pasted and reused; the source code is available here: \href{https://www.overleaf.com/read/hdpkgxgjkbgw}{https://www.overleaf.com/read/hdpkgxgjkbgw}.

    As for the abstract itself, as a general rule: the abstract should not contain more than 200 words. \medskip
    
    \noindent\textbf{Keywords:} Markowitz · Risk Measures · Optimization \par
    \noindent\textbf{Classification:} 91G10 · 91G30 · 91G70
\end{abstract}
\section{Introduction}
Sometimes it is difficult to download stock data or to scrape from finance websites. So its clever to simulate own data, described in Chapter...
This concepts and proofs are based on lecture portfolio optimization by Prof. Pichler. \citet{PichlerPortfoliooptimierung}
\section{Preliminaries}
To obtain the present value $PV$ from a future value $FV$ at time $t>0$ we use the well-known formula $PV = FV e^{-\delta t}$ for some $\delta\in\mathbb{R}$. Hence, the so-called force of interest reads as follows
    \begin{align*}
        \delta = \frac{1}{t}\ln\left(\frac{FV}{PV}\right).
    \end{align*}
In the same way, we design the returns of our assets. For timestamps $\{t_0,\ldots,t_n\}$, the return per time period $[t_{i-1}, t_i]$ of the asset $j\in\{1,\ldots,J\}$ is given by
\begin{equation*}
    \xi_i^j := \frac{1}{t_i-t_{i-1}}\ln\left(\frac{S_{t_i}^j}{S_{t_{i-1}}^j}\right).
\end{equation*}
We store these stock observations in a matrix
\begin{equation*}
    \Xi =
    \begin{pmatrix}
        \xi_0^1 & \cdots & \xi_0^J \\
        \vdots  & \ddots & \vdots  \\
        \xi_n^1 & \cdots & \xi_n^J
    \end{pmatrix}
\end{equation*}
We introduce a probability measure $\pr(\xi = \xi_i^j)\in[0,1]$ for a random variable $\xi\colon\Omega\to\R$.\footnote{$(\Omega,\Sigma,\pr)$ probability space and $(\R,\mathcal{B})$ standard Borel space as usual} We will store these probabilities for every stock $j\in\{1,\ldots,J\}$ in the probability vector
\begin{equation*}
    p =
    \begin{pmatrix}
        p_1 \\
        \vdots \\
        p_n
    \end{pmatrix} =
    \begin{pmatrix}
        \pr(\xi = \xi_1^j) \\
        \vdots \\
        \pr(\xi = \xi_n^j)
    \end{pmatrix} =
    \begin{pmatrix}
        \frac{t_1-t_0}{t_n-t_0} \\
        \vdots \\
        \frac{t_n-t_{n-1}}{t_n-t_0}
    \end{pmatrix}
\end{equation*}
with property $\sum_{i=1}^n\pr(\xi = \xi_i^j) = \sum_{i=1}^n p_i = 1$. Then we obtain after canceling another telescoping sum the expected return for every stock and store them in a vector $r = \E\xi = p^\top\Xi \in\R^J$ with
\begin{equation*}
    \begin{aligned}
    r_j 
    &= \E\xi^j 
    = \sum_{i=1}^n p_i\xi_i^j
    = \sum_{i=1}^n \frac{t_i-t_{i-1}}{t_n-t_0}\frac{1}{t_i-t_{i-1}}\ln\left(\frac{S_{t_i}^j}{S_{t_{i-1}}^j}\right)
    = \frac{1}{t_n-t_0}\sum_{i=1}^n\left(\ln(S_{t_i}^j)-\ln(S_{t_{i-1}}^j)\right) \\
    &= \frac{1}{t_n-t_0}\ln\left(\frac{S_{t_n}^j}{S_{t_0}^j}\right).
    \end{aligned}
\end{equation*}
\section{Simulation of Stocks}
Let $\mu\in\R$ describe the drift and $\sigma\geq0$ represent the volatility of our stock. To simulate a single stock in numerical analysis, we use the Geometric Brownian Motion. It is the solution of the stochastic differential equation
\begin{equation*}
    dS_t = \mu S_tdt + \sigma S_tdW_t
\end{equation*}
where $W_t$ is a Wiener process, also known as Brownian Motion. The equation has the solution
\begin{equation*}
    S_t = S_0\exp\left(\left(\mu-\frac{\sigma^2}{2}\right)t+\sigma W_t\right)
\end{equation*}
for some start stock value $S_0\in\R_{\geq0}$ and this leads to the following algorithm:
\RestyleAlgo{ruled}
\SetKwComment{Comment}{// }{}
\begin{algorithm}
    \caption{Geometric Brownian Motion}
    \label{algo:GBM}
    \KwData{$s_0\in\R_{\geq0}, \mu\in\R, \sigma\geq0$}
    $S_0 \gets s_0$\;
    $i \gets 1$\;
    \While{$i \leq n$}{
        $N_i \gets \mathcal{N}(0,p_i)$ \Comment*[r]{$N_i\sim\mathcal{N}\left(\mu^\prime, \sigma^{\prime2}\right)$ normally distributed}
        $S_i \gets S_{i-1}\exp\left(\left(\mu-\frac{\sigma^2}{2}\right)p_i + \sigma N_i\right)$\;
        $i \gets i+1$\;
    }
    \KwResult{$S\in\R_{\geq0}^n$}
\end{algorithm}
\newpage
\section{Mathematical Models}
\subsection{Markowitz Model}
\begin{theorem}
    Let $r = \E\xi \in\R^J$ be the expected return of the portfolio and $\mu\geq0$ be a lower bound of the expected return $\E x^\top\xi\in\R$ to an allocation $x\in\R^J$. Then the optimization problem
    \begin{equation*}
        \begin{aligned}
            & \underset{x\in\R^J}{\mathrm{minimize}} & & f(x) = x^\top\Sigma x \\
            & \mathrm{subject\,to} & & x^\top r \geq \mu \\
            & & & x^\top\one \leq 1
        \end{aligned}
    \end{equation*}
    has an explicit solution $x^*\colon\R_{\geq 0}\to\R^J$ defined by
    \begin{equation}
    \label{eq:markowitz-solution}
        \begin{aligned}
            x^*(\mu) = \lambda_1^*(\mu)\precision r + \lambda_2^*(\mu)\precision\one
        \end{aligned}
    \end{equation}
    with minimal shadow prices at
    \begin{equation*}
        \begin{aligned}
            & \lambda_1^*(\mu) = \frac{\mu(\one^\top\precision\one) - (r^\top\precision\one)}{(r^\top\precision r)(\one^\top\precision\one) - (r^\top\precision\one)^2}\\
            & \lambda_2^*(\mu) = \frac{(r^\top\precision r) - \mu(r^\top\precision\one)}{(r^\top\precision r)(\one^\top\precision\one) - (r^\top\precision\one)^2}.
        \end{aligned}
    \end{equation*}
\end{theorem}
\begin{proof}
    
\end{proof}
\begin{corollary}
    The solution (\ref{eq:markowitz-solution}) is an affine combination.
\end{corollary}
\begin{proof}
    
\end{proof}
\subsection{Utility Maximization}
\begin{theorem}
    Let $r = \E\xi \in\R^J$ be the expected return of the portfolio and $\kappa>0$ be a measure of risk aversion. Then the optimization problem
    \begin{equation*}
        \begin{aligned}
            & \underset{x\in\R^J}{\mathrm{minimize}} & & g(x) = x^\top r - \frac{\kappa}{2}x^\top\Sigma x \\
            & \mathrm{subject\,to} & & x^\top\one \leq 1
        \end{aligned}
    \end{equation*}
    has an explicit solution $x^*\colon\R_{>0}\to\R^J$ defined by
    \begin{equation}
    \label{eq:utility-maximization}
        \begin{aligned}
            x^*(\kappa) = \frac{1}{\kappa}\precision\left(r+\frac{\kappa-r^\top\precision\one}{\one^\top\precision\one}\one\right).
        \end{aligned}
    \end{equation}
\end{theorem}
\begin{proof}
    
\end{proof}
\begin{corollary}
    The following statements hold true.
    \begin{enumerate}
        \item Solutions (\ref{eq:markowitz-solution}) and (\ref{eq:utility-maximization}) coincide for $\kappa = \frac{\D}{\mu(\C) - (\B)}$ and
        \item Solution (\ref{eq:utility-maximization}) converges to a minimal variance portfolio for $\kappa\to\infty$.
    \end{enumerate}
\end{corollary}
\begin{proof}
    
\end{proof}



\bibliographystyle{abbrvnat}
\label{References}
\bibliography{Literatur}

\end{document}